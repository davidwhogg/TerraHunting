\documentclass[12pt, letterpaper]{article}

\addtolength{\oddsidemargin}{0.25in}
\addtolength{\textwidth}{-0.50in}
\addtolength{\topmargin}{-0.50in}
\addtolength{\textheight}{1.00in}

\begin{document}

Terra Hunting Experiment

Goals and Operational Considerations

Authors: Megan Bedell (Flatiron), David W. Hogg (Flatiron), and others [add your name here]

Executive summary: We consider a set of possible high-level goals for the Terra Hunting Experiment and what they might imply for target selection and survey operations. Etc.

In what follows, we put possible goals in bold, and discuss them.

G1. Find as many planets in [some mass range] in [some period range] around stars in [some spectral range] as we possibly can in a decade of observing.

G2. Some future-discounted version of G1. If we adopt G1 and observe for a full decade before any particular discovery is solid, we open ourselves up to being late to all the relevant parties. So our view is that it is probably better to adopt a goal like G2 over a goal like G1. Interestingly, if we do adopt G2 as a goal, it has non-trivial implications for real-time operations.

G3. Determine the rate at which Sun-like stars host Earth-like planets.

G4. Determine the Earth-like planet occurrence rate as a function of stellar type, age, or metallicity.

G5. Learn everything possible [in one decade] about the planetary systems of 40 particular target stars.

\end{document}
